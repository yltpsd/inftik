\documentclass[12pt,a4paper]{ctexrep}
\setCJKmainfont{AR PL UKai CN}
\setCJKmonofont{AR PL UKai CN}
\title{计算理论与函数式编程简介}
\author{曹竞帆}

\begin{document}
\date{}
\maketitle

\begin{abstract}

本次课主要讨论四个部分的问题。

首先会从数理逻辑的角度讲解计算理论的发源。
通常我们提到计算机发源的时候首先想到的都是ENIAC,
但这是工程实践中的“第一个”。
科学技术的发展从来都是理论先行,那么计算机发源的理论根源在哪里呢?

第二部分我们会开始认识对函数式编程语言产生重要影响,
或者说就是函数式编程语言之理论根基的Lambda演算。
看看两条简单的公理是如何构造出一个完整的计算世界的。

第三部分我们来看看函数式编程语言,思考与“传统”命令式编程语言在
思维方式以及抽象思想上的异同。

最后一部分,我们针对函数式编程语言中的Lisp家族进行剖析,
感受Lisp的优点与不足之处。

最后附上希尔伯特的一句话
\begin{quote}
``We must know---we will know!''
\end{quote}

\end{abstract}

\chapter{希尔伯特的梦}

\section{完备么?一致么?}

\subsection{第三次数学危机}

首先简单交代一下时代背景。
19世纪,伟大的德国数学家康托创立了现代集合论,
虽然没有立即受到其他数学家的认可(康托因此罹患抑郁症并于1918年在精神病院中去世)。
但这并没有阻止后来的数学家将集合论作为基础带入数学的形式化过程中来。
希尔伯特本人就曾经说过:“没有人能够把我们从康托建立的乐园中赶出去。”

然而,早在1897年,福尔蒂就揭示了集合论中的第一个悖论-----布拉利-福尔蒂悖论。

两年后,康托自己也发现了很相似的最大基数悖论。

1901年,罗素又提出了著名的罗素悖论。

由于此时,集合概念已经渗透到众多的数学分支,并且实际上集合论成了数学的基础,
因此集合论中一系列悖论的发现自然地引起了对数学的整个基本结构的有效性的怀疑。

史称“第三次数学危机”。

\subsection{希尔伯特计划}

大卫·希尔伯特,德国著名数学家,一生致力于将数学建立在完美严谨的逻辑的基石上。
1928年,希尔伯特为了捍卫古典数学的尊严,避免悖论以解决数学最基础的问题,
提出了著名的“希尔伯特计划”。

这个计划的主要目标,是为全部的数学提供一个安全的理论基础。具体地,这个基础应该包括:

所有数学的形式化。意思是,所有数学应该用一种统一的严格形式化的语言,并且按照一套严格的规则来使用。

完备性。我们必须证明以下命题:在形式化之后,数学里所有的真命题都可以被证明(根据上述规则)。

一致性。我们必须证明:运用这一套形式化和它的规则,不可能推导出矛盾。

保守性。我们需要证明:如果某个关于“实际物”的结论用到了“假想物”(如不可数集合)来证明,那么不用“假想物”的话我们依然可以证明同样的结论。

判定性。应该有一个算法,来确定每一个形式化的命题是真命题还是假命题。

\subsection{希尔伯特问题}

时间倒回到1900年。

那一年,在巴黎举行的第二届国际数学家大会上,
希尔伯特做了一次堪称数学史上影响最为深远的演讲,
演讲的题目叫做 “数学问题”。
在这一演讲中,希尔伯特列举了二十三个他认为最具重要意义的数学问题。
这些问题被后人称为 “希尔伯特问题”。
解决希尔伯特问题成了许多数学家终生奋斗的目标,
而在解决这些问题的过程中源源不断地产生出的成果则为二十世纪的数学发展注入了极大的生机\cite{tenth}。

其中第二个问题,算数公理系统的相容性问题,于1931年被哥德尔的不完备性定理解决。

随之而来的是希尔伯特计划的第二步无法实行,希尔伯特的梦破灭了。

理论上,哥德尔理论仍留下了一线希望:
也许可以给出一个算法判定一个给定的命题是否是不确定的,
让数学家可以忽略掉这些不确定的命题。

然而,1937年图灵发表他的论文《论可计算数及其在判定问题中的应用》,证明了不存在这样的算法,
希尔伯特计划的最后一步也是不可能的。

尽管这样,哥德尔的不完备性定理仍然带给我们很多教益。
至少我们知道了,有些东西我们不可能知道。
在哥德尔的这个划时代的证明之后,数学家对数学的基本工具-----证明-----有了新的认识。
专门研究数学证明的证明论,在他的启发下蓬勃发展。
但是,哥德尔教给我们最重要的一点是\cite{dream}:

\begin{quote}
数学,如同人生,如同爱情,有些东西是真的,你却永远无法证明。
\end{quote}

\section{可判定么?}

\subsection{形式化可计算函数}


\subsection{图灵的把戏}

\chapter{函数式编程}

\chapter{Why and Why NOT Lisp?}

\begin{thebibliography}{99}
\bibitem{face} 玑衡,2012.面对面的办公室-----纪念艾伦•图灵百年诞辰.豆瓣网
\bibitem{tenth} 卢昌海,2005.Hilbert第十问题漫谈(上).卢昌海的个人主页
\bibitem{dream} 方弦,2009.希尔伯特之梦,以及梦的破灭.科学松鼠会
\end{thebibliography}

\end{document}
